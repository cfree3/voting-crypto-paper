% proposal.tex | CS 6260 - "Vendetta" Proposal

% setup
\documentclass[10pt]{article}
\usepackage{fullpage}
\usepackage{enumerate}
\usepackage{hyperref}
\usepackage{multicol}

% don't number sections
\setcounter{secnumdepth}{0}

% don't highlight links
\hypersetup{hidelinks}

% custom commands
\newcommand{\todo}[1]{\textcolor{red}{\textbf{TODO:} \emph{#1}}}
\newcommand{\term}[1]{\textit{#1}}
\newcommand{\bterm}[1]{\textbf{\textit{#1}}}
\newcommand{\code}[1]{\texttt{#1}}
\newcommand{\super}[1]{\textsuperscript{#1}}
\newcommand{\sub}[1]{\textsubscript{#1}}

% title
\title{Cryptographic assurance of fairness in voting}
\date{}
\author{
\begin{tabular}{c c c}
Kelsey Francis & Curtis Free & Christopher Martin \\
\small \tt{francis@gatech.edu} & \small \tt{curtis.free@gatech.edu} & \small \tt{chris.martin@gatech.edu}
\end{tabular}
}

% document
\begin{document}
\maketitle

\thispagestyle{empty}

\begin{multicols}{2}

\section{Background}

Citizens' ability to trust the fairness of elections is vital to governments whose rule
is subject to the consent of the governed, but it is not uncommon for the veracity of
election results to come into question.
Current election processes do not allow participants to verify the accuracy of vote counting.

Problems such as identity fraud and equality of ballot access are not amenable to
cryptographic solutions, but their propensity to be externally observable limits the
potential for widespread abuse.
Internal manipulation by the government conducting the poll is a more insidious threat,
and it is here where we may fruitfully apply cryptography to ensure honesty.
Research in this field has yielded a number of candidate systems to replace or accompany
existing voting procedures to provide fairness guarantees while maintaining ballot secrecy.

% But how often is this fairness challenged? Given voting systems widely used today, citizens
% must trust that the government (a) counts all cast votes (correctly) and (b) counts
% \emph{only} votes actually cast.

% Researchers have proposed a variety of systems that make use of cryptography to empower the
% people to verify in an election that one or both of those requirements has been met by the
% body performing the election -- with the crucial stipulation that this verification allows
% individual ballots to remain secret.

\section{Security goals}

We will formulate a precise security definitions of requirements we believe
must be met  in a ``fair'' and ``private'' election.

Fairness requires that enough information is published such that anyone is able to
independently tabulate results, and a receipt given to each voter provides evidence
of fraud in the event that the voter's ballot is miscounted.

Ballot secrecy is a constraint that most cryptography applications do not require.
When votes are counted correctly, a voter must not be able to prove any
facts about the ballot that was cast.

\section{Literature survey}

We will then study existing literature on the subject, paying close attention to existing
end-to-end auditable voting systems.
Our primary focus is in-person systems such as Scantegrity \cite{scantegrity} and
Pr\^{e}t \`{a} Voter \cite{preta}.
We also consider online voting systems such as Helios \cite{helios}.

\section{Proof of concept}

If time permits, we will construct a proof-of-concept implementation of a system that meets
our goals, based on the systems described in the literature if they are sufficient.

\bibliography{sources}{}
\bibliographystyle{plain}

\end{multicols}

\end{document}
