% proposal.tex | CS 6260 - "Vendetta" Proposal

% setup
\documentclass[10pt]{article}
\usepackage{amsmath}
\usepackage{amssymb}
\usepackage{fullpage}
\usepackage{color}
\usepackage{enumerate}
\usepackage{hyperref}
\usepackage{mathtools}
\usepackage{listings}
\usepackage{array}
\usepackage{graphicx}
\usepackage{parskip}
\usepackage{multicol}

% don't number sections
\setcounter{secnumdepth}{0}

% don't highlight links
\hypersetup{hidelinks}

% custom commands
\newcommand{\todo}[1]{\textcolor{red}{\textbf{TODO:} \emph{#1}}}
\newcommand{\term}[1]{\textit{#1}}
\newcommand{\bterm}[1]{\textbf{\textit{#1}}}
\newcommand{\code}[1]{\texttt{#1}}
\newcommand{\super}[1]{\textsuperscript{#1}}
\newcommand{\sub}[1]{\textsubscript{#1}}

% title
\title{Cryptographic techniques to ensure fair elections}
\date{}
\author{
\begin{tabular}{c c c}
Kelsey Francis & Curtis Free & Christopher Martin \\
\small \tt{francis@gatech.edu} & \small \tt{curtis.free@gatech.edu} & \small \tt{chris.martin@gatech.edu}
\end{tabular}
}

% document
\begin{document}
	\maketitle

	\begin{abstract}
		Current election processes do not allow participants to verify the accuracy of vote counting.
		Applied cryptography research has yielded a number of candidate systems to replace or accompany
		voting existing voting procedures to provide fairness guarantees while maintaining ballot secrecy.
		We will define a set of security properties sufficient to make an election both ``fair'' and ``secret'',
		survey existing literature on the matter, and -- if time permits -- produce a proof-of-concept
		implementation of a system that meets these goals (either the best system we find, or one of
		our own creation using borrowed techniques).
	\end{abstract}
	\begin{multicols}{2}

		\section{Background}

			Fairness of elections is of vital importance to nations such as the United States in which
			government officials are elected to office by the people.

			But how often is this fairness challenged? Given voting systems widely used today, citizens
			must trust that the government (a) counts all cast votes (correctly) and (b) counts
			\emph{only} votes actually cast.

			Researchers have proposed a variety of systems that make use of cryptography to empower the
			people to verify in an election that one or both of those requirements has been met by the
			body performing the election -- with the crucial stipulation that this verification allows
			individual ballots to remain secret.

		\section{Proposal}

			Our project will accomplish three goals.

			\subsection{Specification of security goals}

				We plan to begin by defining what requirements we believe must be met in order for an
				election to be considered ``fair'' and translating those requirements into precise security
				goals that we can reason about using cryptographic techniques. (For example, at what points
				in the election should we require authenticated encryption, and what are the appropriate
				unforgeability constraints?)

			\subsection{Literature survey}

				We will then study existing literature on the subject, paying close attention to existing
				end-to-end auditable voting systems, including both in-person systems (our primary focus)
				such as Scantegrity \cite{scantegrity} and Pr\^{e}t \`{a} Voter \cite{preta} and online
				voting systems that attempt to meet the same goals such as Helios \cite{helios}.

			\subsection{Proof-of-concept implementation}

				Having surveyed prior research, and time-permitting, we will consider the extent to which
				previously-developed systems meet our requirements and construct a proof-of-concept
				implementation of one such system or one of our own design, using borrowed techniques in
				an attempt to meet all the goals we set forth.

		\bibliography{sources}{}
		\bibliographystyle{plain}

	\end{multicols}

\end{document}

% vim: set colorcolumn=101 cursorcolumn cursorline:

