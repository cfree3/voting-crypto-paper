% proposal.tex | CS 6260 - "Vendetta" Proposal

% setup
\documentclass[10pt]{article}
\usepackage{amsmath}
\usepackage{amssymb}
\usepackage{fullpage}
\usepackage{color}
\usepackage{enumerate}
\usepackage{hyperref}
\usepackage{mathtools}
\usepackage{listings}
\usepackage{array}
\usepackage{graphicx}
\usepackage{parskip}
\usepackage{multicol}

% don't number sections
\setcounter{secnumdepth}{0}

% don't highlight links
\hypersetup{hidelinks}

% custom commands
\newcommand{\todo}[1]{\textcolor{red}{\textbf{TODO:} \emph{#1}}}
\newcommand{\term}[1]{\textit{#1}}
\newcommand{\bterm}[1]{\textbf{\textit{#1}}}
\newcommand{\code}[1]{\texttt{#1}}
\newcommand{\super}[1]{\textsuperscript{#1}}
\newcommand{\sub}[1]{\textsubscript{#1}}

% title
\title{``Vendetta'': End-to-end auditable voting}
\date{}
\author{Kelsey Francis, Curtis Free, Christopher Martin}

% document
\begin{document}
	\maketitle

	\begin{abstract}
		\todo{Write the abstact.}
	\end{abstract}

	\begin{multicols}{2}

		\section{Background}

			A democratic voting system like that in the United States raises several questions to which
			cryptography can potentially provide answers. Namely, the question at hand is \emph{how to keep
			voting fair}. In particular, we want to consider the following questions:
			\begin{itemize}

				\item
					How can a government prove to a voter that his/her vote \emph{was} counted (correctly),
					without being able to reveal the candidate for which the ballot was cast?

				\item
					How can an observer verify that the votes reported by a state contain no forged votes (i.e.,
					that the state is not cheating by introducing votes not actually cast)?

			\end{itemize}
			Cryptographic systems designed to meet exactly these goals are called \term{end-to-end auditable
			voting systems}\cite{end_to_end}.

			A wealth of research has been done in this area, and a number of such systems have been devised.
			Most of these systems have been developed by academia. However, countries such as Estonia and
			even the United States have developed systems for voting via the Internet
			\cite{estonia,us_vs_est} -- and while that is only tangential to the questions posed above, it
			can be instructive to see how (if at all) these systems meet those goals.

		\section{Proposal}

			We plan to begin with a more precise quantification of the security goals that we believe must
			be met for a voting system (whether physical or digital) in order to accomplish the higher-level
			goals questioned above. We will then survey existing research on such cryptographic systems to
			determine whether any fully meet the security requirements.

			Should an existing system sufficiently meet our requirements, we plan to construct a
			proof-of-concept implementation of the system along with an analysis both of its effectiveness
			at ensuring fairness and of the feasibility of employing such a system in the real world.

			Otherwise, we will examine the flaws and strengths of the existing designs and attempt to
			construct a sufficient system -- or in the least, one that shows a marked improvement over
			prior work. We will then provide an analysis of how our proposed \emph{new} system meets the
			stated goals and could prove useful in a real-world deployment. Time permitting, we will also
			implement a proof-of-concept for our scheme.

	\end{multicols}

	\bibliography{sources}{}
	\bibliographystyle{plain}
\end{document}

% vim: set colorcolumn=101 cursorcolumn cursorline:

